\documentclass[12pt, a4paper]{article}

\usepackage[french]{babel}
\usepackage[utf8]{inputenc}
\usepackage{amsfonts,vmargin}
\usepackage{amsmath}
\usepackage{amssymb}

\begin{document}

\begin{center}
\large\sc 

TD : 1 mathématiques : Nombres complexes
\end{center}

{\bf Exercice~1}~: Ecrire sous la forme $a+bi$ les nombres complexes suivants~:

\begin{equation*}
Z_{1}=\frac{2-i\sqrt{3}}{\sqrt{3}-2i},~Z_{2}=\frac{(2+i)(3+2i)}{2(2-i)},~Z_{3} = \frac{3+4i}{(2+3i)(4+i)},~Z_{4}=[\frac{2+i^5}{1+i^{15}}]^2
\end{equation*}

{\bf Exercice~2}~: Déterminer le paramètre $\alpha$ pour que~:
\begin{equation*}
Z=\frac{1+\alpha i}{2\alpha + i(\alpha^2-1)}
\end{equation*}
soit purement imaginaire.

{\bf Exercice~3}~: Déterminer les modules et les arguments des nombres complexes suivants :
\begin{equation*}
Z_{1}=1+i,~Z_{2}=\frac{1-i}{1+i},~Z_{3}=\frac{1+i\sqrt{3}}{\sqrt{3}+i},~Z_{4}=(1+i)^8(1-i\sqrt{3})^{-6}
\end{equation*}

{\bf Exercice~4}~: Simplifier les expressions suivantes :
\begin{equation*}
Z_{1}=\frac{\cos\alpha+i\sin\alpha}{\cos\beta-i\sin\beta},~Z_{2}\frac{(1-i\sqrt{3})(\cos\alpha+i\sin\alpha)}{2(1-i)(\cos\alpha-i\sin\alpha)}
\end{equation*}

{\bf Exercice~5}~: Déterminer les parties réelles et les parties imaginaires des nombres complexes suivants :
\begin{equation*}
Z_{1} = \exp(-1+\frac{i\pi}{6}),~Z_{2}=\exp(2-i),~Z_{3}=\exp(\frac{-i\pi}{2}),~Z_{4}=\exp(1+i)(-2+\frac{i\pi}{3})
\end{equation*}

{\bf Exercice~6}~: 
a) Résoudre dans $\mathbb{C}$ les équations suivantes :

\begin{equation*}
Z^2+Z\sqrt{3}+1=0,~Z^2=-8+6i
\end{equation*}
b) Déduire les solutions dans $\mathbb{C}$ de l'équation:
\begin{equation*}
Z^2+(-3+i)Z+4-3i=0
\end{equation*}

{\bf Exercice~7}~: Résoudre dans $\mathbb{C}$ l'équation : 
$Z^2+(1-5i)Z-3i-6=0$ sachant que l'une des solutions est imaginaire pure.

{\bf Exercice~8}~:\\ 1- Sachant que Z appartient au cercle trigonométrique unitaire $(z = e^{i\theta})$, simplifier l'expression du complexe : $Z = \frac{1-z}{1+z}$.
\\
2- Simplifier l'expression : $A = (1+Z)^n$ avec $Z=e^\frac{2i\pi}{3}$ avec $n\in\mathbb{C}$.

{\bf Exercice~9}~: Calculer les racines carrées des nombres complexes suivants :
\begin{equation*}
Z_{1}=2i,~Z_{2}=\frac{1+i}{1-i},~Z_{3}=1+i
\end{equation*}

{\bf Exercice~10}~: Calculer les racines cubiques des nombres complexes suivants :
\begin{equation*}
Z_{1}=1,~Z_{2}=i,~Z_{3}=2-2i,~Z_{4}=\frac{\sqrt{3}+i}{-\sqrt{3}+i}
\end{equation*}


\end{document}
