\documentclass[12pt, a4paper]{article}

\usepackage[french]{babel}
\usepackage[utf8]{inputenc}
\usepackage{amsfonts,vmargin}
\usepackage{amsmath}
\usepackage{amssymb}

\begin{document}

\begin{center}
\large\sc 
Rappels de cours : nombres complexes
\end{center}
Soit le nombre complexe $Z$, sa forme algébrique est donnée :
\begin{equation*}
Z = a + bi~(a \in \mathbb{R}, ~ b \in \mathbb{R} ~ et ~ i^2=-1)
\end{equation*}
$a$ est la partie réelle du nombre imaginaire $Z$ : $Re(Z)=a$.
\\
$b$ est la partie imaginaire du nombre complexe $Z$ : $Im(Z)=b$.
\\
\\
$Z$ est dit purement imaginaire si et seulement si $a=0$.
\\
$Z$ est dit purement réel si et seulement si $b=0$.
\\
\\
Le module du nombre complexe $Z$ est la distance OM tel que : $r=|Z|=\sqrt{a^2+b^2}$.\\
L'argument du nombre complexe $Z$ est l'angle $\theta$ tel que : $Arg(Z)=\theta$.
\\
\\
Z peut s'écrire sous la forme Trigonométrique suivante :
\begin{equation*}
Z = r(\cos \theta+i\sin \theta)
\end{equation*}
Ou encore sous la forme Exponentielle :
\begin{equation*}
Z = re^{i\theta}
\end{equation*}
{\bf Conjugué d'un nombre complexe}
\\
Le nombre complexe conjugué de $Z=a+bi$ est le nombre complexe $\overline{Z}=a-bi$.
\begin{equation*}
Z\overline{Z} = (a+bi)(a-bi) = a^2+b^2 = |Z|^2= |\overline{Z}|^2
\end{equation*}\\
Soit $Z_{1} = a_{1}+b_{1}i$ et $Z_{2} = a_{2}+b_{2}i$ deux nombres complexes.
\begin{equation*}
\overline{Z_{1}+Z_{2}}=\overline{Z_{1}}+\overline{Z_{2}}=(a_{1}+a_{2})-i(b_{1}+b_{2})
\end{equation*}
\begin{equation*}
\overline{Z_{1} Z_{2}}=\overline{Z_{1}}~\overline{Z_{2}}=(a_{1}a_{2}-b_{1}b_{2})-i(a_{1}b_{2}+a_{2}b_{1})
\end{equation*}
{\bf Propriétés des nombres complexes}
\\
Soit deux nombres complexes $Z_{1} = a_{1}+b_{1}i$ et $Z_{2} = a_{2}+b_{2}i$.
\\
Égalité : $Z_{1} = Z_{2}~\Rightarrow~a_{1} = a_{2}~et~b_{1} = b_{2}$.
\\
Produit : ${Z_{1}Z_{2}}=(a_{1}a_{2}-b_{1}b_{2})+i(a_{1}b_{2}+a_{2}b_{1})$
\\
$|Z_{1}Z_{2}| = |Z_{1}| |Z_{2}|$
\\
$Arg(Z_{1}Z_{2}) = Arg(Z_{1}) +  Arg(Z_{2})$
\\
$Z_{1}Z_{2} = r_{1}r_{2} ( cos(\theta_{1}+\theta_{2})+i(\sin(\theta_{1}+\theta_{2})) 
= r_{1}r_{2}e^i(\theta_{1}+\theta_{2})$
\\
Quotient :
\\
\begin{equation*}
\frac{Z_{1}}{Z_{2}}
=\frac{a_{1}+b_{1}i}{a_{2}+b_{2}i}=\frac{(a_{1}+b_{1}i)( a_{2}-b_{2}i)}{(a_{2}+b_{2}i)(a_{2}-b_{2}i)}
=\frac{(a_{1}+b_{1}i)( a_{2}-b_{2}i)}{a_{2}^2+b_{2}^2} 
= \frac{Z_{1}\overline{Z_{2}}}{|Z_{2}|^2}
\end{equation*}
\begin{equation*}
|\frac{Z_{1}}{Z_{2}}| = \frac{|Z_{1}|}{|Z_{2}|}
\end{equation*}
\begin{equation*}
Arg(\frac{Z_{1}}{Z_{2}}) = Arg(Z_{1})-Arg(Z_{2})
\end{equation*}

\begin{equation*}
\frac{Z_{1}}{Z_{2}}= \frac{r_{1}}{r_{2}}( cos(\theta_{1}-\theta_{2})+i(\sin(\theta_{1}-\theta_{2})) = \frac{r_{1}}{r_{2}}e^{i(\theta_{1}-\theta_{2})}
\end{equation*}


\begin{center}
\large\sc 
TD : 1 mathématiques : Nombres complexes
\end{center}

{\bf Exercice~1}~: Ecrire sous la forme $a+bi$ les nombres complexes suivants~:

\begin{equation*}
Z_{1}=\frac{2-i\sqrt{3}}{\sqrt{3}-2i},~Z_{2}=\frac{(2+i)(3+2i)}{2(2-i)},~Z_{3} = \frac{3+4i}{(2+3i)(4+i)},~Z_{4}=[\frac{2+i^5}{1+i^{15}}]^2
\end{equation*}

{\bf Exercice~2}~: Déterminer le paramètre $\alpha$ pour que~:
\begin{equation*}
Z=\frac{1+\alpha i}{2\alpha + i(\alpha^2-1)}
\end{equation*}
soit purement imaginaire.

{\bf Exercice~3}~: Déterminer les modules et les arguments des nombres complexes suivants :
\begin{equation*}
Z_{1}=1+i,~Z_{2}=\frac{1-i}{1+i},~Z_{3}=\frac{1+i\sqrt{3}}{\sqrt{3}+i},~Z_{4}=(1+i)^8(1-i\sqrt{3})^{-6}
\end{equation*}

{\bf Exercice~4}~: Simplifier les expressions suivantes :
\begin{equation*}
Z_{1}=\frac{\cos\alpha+i\sin\alpha}{\cos\beta-i\sin\beta},~Z_{2}=\frac{(1-i\sqrt{3})(\cos\alpha+i\sin\alpha)}{2(1-i)(\cos\alpha-i\sin\alpha)}
\end{equation*}

{\bf Exercice~5}~: Déterminer les parties réelles et les parties imaginaires des nombres complexes suivants :
\begin{equation*}
Z_{1} = e^{-1+\frac{i\pi}{6}},~Z_{2}=e^{2-i},~Z_{3}=e^{\frac{-i\pi}{2}},~Z_{4}=e^{1+i}e^{-2+\frac{i\pi}{3}}
\end{equation*}

{\bf Exercice~6}~: \\
a) Résoudre dans $\mathbb{C}$ les équations suivantes :

\begin{equation*}
Z^2+Z\sqrt{3}+1=0,~Z^2=-8+6i
\end{equation*}
b) Déduire les solutions dans $\mathbb{C}$ de l'équation:
\begin{equation*}
Z^2+(-3+i)Z+4-3i=0
\end{equation*}

{\bf Exercice~7}~: Résoudre dans $\mathbb{C}$ l'équation : 
$Z^2+(1-5i)Z-3i-6=0$ sachant que l'une des solutions est imaginaire pure.

{\bf Exercice~8}~:\\ 1- Sachant que Z appartient au cercle trigonométrique unitaire $(z = e^{i\theta})$, simplifier l'expression du complexe : $Z = \frac{1-z}{1+z}$.
\\
2- Simplifier l'expression : $A = (1+Z)^n$ avec $Z=e^\frac{2i\pi}{3}$ avec $n\in\mathbb{N}$.

{\bf Exercice~9}~: Calculer les racines carrées des nombres complexes suivants :
\begin{equation*}
Z_{1}=2i,~Z_{2}=\frac{1+i}{1-i},~Z_{3}=1+i
\end{equation*}

{\bf Exercice~10}~: Calculer les racines cubiques des nombres complexes suivants :
\begin{equation*}
Z_{1}=1,~Z_{2}=i,~Z_{3}=2-2i,~Z_{4}=\frac{\sqrt{3}+i}{-\sqrt{3}+i}
\end{equation*}


\end{document}
