\documentclass[12pt, a4paper]{article}

\usepackage[french]{babel}
\usepackage[utf8]{inputenc}
\usepackage{amsfonts,vmargin}
\usepackage{amsmath}
\usepackage{amssymb}

\begin{document}

\begin{center}
\large\sc 

TD : 3 mathématiques : Polynômes et fractions rationnelles
\end{center}

{\bf Exercice~1}~: Soit le polynôme suivant~:

\begin{equation*}
P(x) = x^4-5x^3+13x^2-19x+10
\end{equation*}
1- Calculer $P(1)$ et $P(2)$. \\
2- Déduire la factorisation du polynôme $P(x)$ dans $\mathbb{R}$.

{\bf Exercice~2}~:
\begin{equation*}
P(z) = z^4+z^3-z^2+6
\end{equation*}
1- Calculer $P(1+i)$. \\
2- Déduire les solutions de $P(z)$.

{\bf Exercice~3}~: Soit P un polynôme. Le reste de la division de $P$ par $x-1$ est $3$, et celui de $P$ par $x-2$ est $2$. \\
Quel est le reste de la division de $P$ par $x^3-3x+2$?

{\bf Exercice~4}~: Effectuer la division suivant les puissances décroissantes du polynômes $A(x)=x^3+2x^2-5$ par le polynôme $B(x)=x-1$. \\
Trouver les quartes coefficients réels a,b,c et d de la décomposition en élements simples de la fonction rationnelles $F$:

\begin{equation*}
F(x)=\frac{A(x)}{B(x)}=\frac{x^3+2x^2-5}{x-1}=ax^2+bx+c+\frac{d}{x-1}
\end{equation*}

{\bf Exercice~5}~: Décomposer les fraction rationnelles suivantes en éléments simples (par identification des coefficients et substitution)

\begin{equation*}
f(x)=\frac{1-2x}{(x^2+1)(x+2)^2}
\end{equation*}
\begin{equation*}
f(x)=\frac{x^2+1}{x^4+x^3-x-1}
\end{equation*}

{\bf Exercice~6}~: Déterminer l'ordre de multiplicité de la racine 1 du polynôme 
\begin{equation*}
P(x)=x^5-5x^4+14x^3-22x^2+17x-5
\end{equation*}

{\bf Exercice~7}~: Décomposer en éléments simples sur le corps des nombres réels les fonctions suivantes :

\begin{equation*}
f_{1}(x) = \frac{x}{(x-1)^3(x-2)}
\end{equation*}
\begin{equation*}
f_{2}(x) = \frac{2x+1}{(x^2-1)^3}
\end{equation*}

\end{document}
